\chapter{Language description}\label{language}

The language implemented by the compiler is a subset of JavaScript with added type annotations and slightly modified semantics of arithmetic operations. In a way, this makes it a subset of a modified TypeScript.

The version of JavaScript used as the base is standardized in the ECMAScript 2023 Language Specification (ECMA-262)~\cite{ecma262} -- this document serves as the primary reference for the language implemented by the compiler.

TypeScript, on the other hand, does currently not have a formal standard and relies on an official reference implementation. The last version with an available specification was TypeScript 1.8~\cite{typescript18}. This version serves as a reference for the extended grammar of our language.

While this chapter together with the ES specification describe the full language we would like to implement, only a part of this language is supported by the compiler. The language features that are currently supported are described later in Chapter \ref{chap:supported_lang}.


\section{Grammar}\label{lang:grammar}

The language grammar extends the grammar defined in ECMA-262 by allowing type annotations in function signatures and variable declarations. Grammar productions in this text are described using the same notation as is used in the specification. The notation is described in Section 5.1.5 of the specification \cite{ecma262}.

A type annotation is defined by the following production:

\GrammarRule[TypeAnnotation]{}{
    \terminal{:} \nonterminal[Identifier]{}{}
}


The declarations of variables and type annotations then follow the modified productions below (with added \nonterminal[TypeAnnotation]{}{} non-terminal being the only change):


\GrammarRule[VariableDeclaration]{In, Yield, Await}{
    \nonterminal[BindingIdentifier]{?Yield, ?Await}{} \nonterminal[TypeAnnotation]{}{opt} \nonterminal[Initializer]{}{opt} \\
    \nonterminal[BindingPattern]{?Yield, ?Await}{} \nonterminal[TypeAnnotation]{}{opt} \nonterminal[Initializer]{}{}
}
\GrammarRule[LexicalBinding]{In, Yield, Await}{
    \nonterminal[BindingIdentifier]{?Yield, ?Await}{} \nonterminal[TypeAnnotation]{}{opt} \nonterminal[Initializer]{}{opt} \\
    \nonterminal[BindingPattern]{?Yield, ?Await}{} \nonterminal[TypeAnnotation]{}{opt} \nonterminal[Initializer]{}{}
}
\GrammarRule[SingleNameBinding]{In, Yield, Await}{
    \nonterminal[BindingIdentifier]{?Yield, ?Await}{} \nonterminal[TypeAnnotation]{}{opt} \nonterminal[Initializer]{}{opt}
}

Because context-free grammars are not powerful enough to represent the set of all valid JavaScript programs, the language also defines so-called \textit{static semantics}. These are additional rules associated with grammar productions and usually define an algorithm which decides whether the input is valid and can be evaluated.

An example of a static semantics rule may be the check whether a left-hand side expression of an assignment is a valid assignment target.


\section{Value types}\label{lang:types}

There are several types available in JavaScript: \texttt{Undefined}, \texttt{Null}, \texttt{Boolean}, \texttt{String}, \texttt{Symbol}, \texttt{Number}, \texttt{BigInt}, and \texttt{Object}.

All these types, except \texttt{Object}, are primitive types. Primitive types are immutable and represent the simplest values available in JavaScript. The types \texttt{Undefined} and \texttt{Null} both have a single value -- referred to as \texttt{undefined} and \texttt{null} respectively.

The \texttt{Object} type represents a collection of properties. Each property is identified by a \texttt{String} or \texttt{Symbol} value, and its value can be of any type. An object is generally mutable, and its properties can be added, removed, and rebound. Functions and class constructors are a special, callable, kind of objects.

The \texttt{Number} type is a double-precision 64-bit format IEEE 754-2019~\cite{ieee754} floating point number. The \texttt{BigInt} type is a signed integer of arbitrary precision.

Aside from the standard types, we define a new primitive type \texttt{Int32} which guarantees that all arithmetic operations with two \texttt{Int32} operands will result in a new value of the same type. The type represents a 32-bit two's complement signed integer. Operations with the type follow the usual integer arithmetic with overflow. The \texttt{Int32} type can be used in any other context where the specification expects a value of the type \texttt{Number}. In these cases, the value behaves identical to an equivalent value of type \texttt{Number}.

A new type \texttt{Float64} is also defined. This type has the same semantics as the \texttt{Number} type, but tells the compiler that the value should always be represented as a floating point number. Such values are easier to manipulate and operations can be faster if the values are expected not to take advantage of an optimized \texttt{Number} type.

The reason for adding these new types is that many microcontrollers do not have a 64-bit FPU (floating-point unit). It is therefore desirable to avoid floating point operations whenever possible. While many interpreters, including QuickJS, use an integer type to represent some values of the \texttt{Number} type as an optimization, when the result of an operation does not fit in this type, the operands have to be promoted to a floating point number and the operation has to be performed in the floating point domain. Even if a program is written so that no floating point operations are performed, this approach still has two downsides:
\begin{itemize}
    \item every arithmetic operation contains an overflow check, and
    \item the \texttt{Number} type is a union of two types.
\end{itemize}

\noindent
For the use in type annotations, the following type names are allowed:
\begin{itemize}
    \item \texttt{int32} -- 32-bit signed integer,
    \item \texttt{float64} -- 64-bit floating point number,
    \item \texttt{boolean} -- boolean value,
    \item \texttt{object} -- object reference\footnote{Unlike in standard TypeScript, the type annotation \texttt{object} does not allow assignment of values \texttt{null} and \texttt{undefined}.},
    \item \texttt{any} -- any value (dynamic type -- may contain any value, including those not listed above like \texttt{undefined}),
    \item \texttt{void} -- no return value (in JavaScript treated as \texttt{undefined}).
\end{itemize}

While the list is quite limited and not well usable for checking the correct usage of more complex types, it is sufficient for allowing the compiler to perform some optimizations of the usage of primitive types. Extending the type system to support more complex types would require a lot of effort, which is out of the scope of this thesis. It is, however, a consideration for future work.


\section{Language semantics}\label{lang:semantics}

The semantics of our language are the same as described in ECMA-262~\cite{ecma262} except for the cases described below.

Because variables, and bindings in general, in JavaScript are dynamically typed, many operations with them start with a type check and the selection of correct behavior of the operation. With statically typed bindings, however, we can perform this selection at compile time and speed up program execution. It is therefore advantageous to define these operations in such a way that the result type is always the same independently on the input values. We can also often replace the typical tagged union, which is used to represent a JavaScript value, with only a value without the tag.

New implicit conversions are defined for the types \texttt{Int32} and \texttt{Float64}. When an arithmetic operation, except exponentiation, is performed, the following rules are applied:

\begin{itemize}
    \item If both operands are of the type \texttt{Int32}, no conversion is performed and the operation is performed using 32-bit integer arithmetic with overflow.
    \item If one of the operands is of the type \texttt{Float64} and the other is either \texttt{Int32} or \texttt{Float64}, the operation is performed in the floating point domain. The eventual \texttt{Int32} operand is first converted to \texttt{Float64}.
    \item If one of the operands is of a different type than \texttt{Int32} or \texttt{Float64}, the operation is performed by the standard JavaScript rules. In this case, operands of the types \texttt{Int32} and \texttt{Float64} are converted to the type \texttt{Number}.
\end{itemize}

In other contexts where the types \texttt{Int32} and \texttt{Float64} are not directly supported, the values are first converted to the type \texttt{Number} and the operation is performed using standard JavaScript semantics\footnote{While the behavior of these types is described using a conversion, it is expected that the values are only treated in the same way, and an optimization is performed.}.

Because bindings are statically typed, when an assignment is performed, the right-hand side operand is first converted to the type of the binding and only then is the binding changed. The same conversion is also performed with return values and call operands. In the case of some conversions, the exact value may not be changed (e.g., from \texttt{Int32} to \texttt{Any}), while in some cases the result may be different (e.g., from \texttt{Float64} to \texttt{Int32}).
