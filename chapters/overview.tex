\chapter{Related projects}

This chapter describes other projects used for the implementation of the proposed compiler. It also describes multiple small compiler backends which were considered as candidate options.

\section{Jaculus}

Jaculus is a platform for programming microcontrollers using JavaScript. It uses the QuickJS JavaScript engine. Jaculus' core library Jaculus-machine is of particular interest for this work, as the library is used for implementing the compiler. Jaculus-machine implements high-level C++ abstractions around QuickJS and implements some core features for the runtime.

\section{QuickJS}

QuickJS is a JavaScript engine. The version used in the current version of Jaculus-machine implements the ECMAScript 2020 Language Specification\cite{ecma262}.

QuickJS is designed to have a small code and memory footprint, and to be easily embeddable in larger programs.

QuickJS uses a stack-based bytecode machine. Values in JavaScript have a dynamic type, which may fall in several categories. In QuickJS, values are represented using a tagged union - a structure with two 64-bit values representing the tag and a corresponding value.

Some primitive values are placed directly into this structure while others are allocated on the heap and the structure only contains a pointer to the value. Values which require memory allocation use reference counting with cycle detection for memory management.

\todo{expand}


\section{Compiler backends}

\todo{write}

\subsection{MIR}

MIR is a small compiler backend designed for usage in JIT scenarios.
