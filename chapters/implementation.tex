\chapter{Implementation}

The compilation pipeline consists of multiple phases. Its input is the content of a single TypeScript file, and its output is compliant JavaScript code and compiled functions. These compiled functions are made available to the JavaScript runtime as stubs.

The compiler first parses the input source code and generates an abstract syntax tree (AST). Selected functions from the AST are then transformed into our intermediate representation (IR). MIR code is then generated from the IR. The MIR code is then processed by the MIR backend which generates machine code for the target architecture.


\section{Parser}

The first phase of the compilation pipeline involves parsing the input source code. The TypeScript code is converted to an abstract syntax tree (AST) using a recursive descent parser. The design of the parser is inspired by the parser described in the book \textit{Crafting Interpreters}~\cite{craftinginterpreters}. The parser includes a tokenization step which converts the input source code into a stream of tokens, which are then processed using recursive descent.

Recursive descent is a technique for constructing an LL parser which uses a top-down approach. It can be implemented by transforming each grammar non-terminal into a function. When a function for a non-terminal is called, it tries to match the input against expansion rules by recursively calling the functions for other non-terminals.

While recursive descent parsers are easy to implement and understand, they are not able to parse left-recursive grammars. Because the grammar of our language, as described in Section \ref{lang:grammar}, is left-recursive, slight modifications to the grammar and the algorithm are needed. The changes are described further in this section.

Infix binary expressions are a case of left recursion in the grammar. They are instead parsed using a cover grammar\footnote{A cover grammar is a more general grammar which is easier to parse but may not produce an equivalent parse tree. The cover grammar is then refined into the correct tree using a separate algorithm.} which is later refined into the correct expression tree. Because the grammar notation from the specification cannot describe iteration, the cover grammar is described using EBNF~\cite{iso14977}:

\begin{minted}{ebnf}
BinaryExpression = UnaryExpression,
                   { InfixBinaryOperator, UnaryExpression };
\end{minted}

The result is a flat representation of the expression without regard to operator precedence and associativity. It is then refined to the correct expression tree using Shunting Yard algorithm~\cite{algol60}.

Another cases of left recursion in the grammar are the non-terminals \nonterminal[MemberExpression]{}{} and \nonterminal[CallExpression]{}{}. In both cases, there are multiple productions with left recursion. Fortunately, all of these productions represent a repeated application of a rule. Therefore, they can be parsed iteratively and then transformed into a correct AST. The non-left recursive productions are always applied first, and the left recursive productions are applied in a loop until no more productions can be applied. The grammar for these two non-terminals is defined as follows:

\GrammarRule[MemberExpression]{Yield, Await}{
    \nonterminal[SuperProperty]{?Yield, ?Await}{} \\
    \nonterminal[MetaProperty]{}{} \\
    \nonterminal[PrimaryExpression]{?Yield, ?Await}{} \\
    \terminal{new} \nonterminal[MemberExpression]{?Yield, ?Await}{} \nonterminal[Arguments]{}{} \\
    \nonterminal[MemberExpression]{?Yield, ?Await}{} \terminal{[} \nonterminal[Expression]{?Yield, ?Await}{} \terminal{]} \\
    \nonterminal[MemberExpression]{?Yield, ?Await}{} \terminal{.} \nonterminal[IdentifierName]{}{} \\
    \nonterminal[MemberExpression]{?Yield, ?Await}{} \terminal{.} \nonterminal[PrivateIdentifier]{}{} \\
    \nonterminal[MemberExpression]{?Yield, ?Await}{} \nonterminal[TemplateLiteral]{}{}
}

\GrammarRule[CallExpression]{Yield, Await}{
    \nonterminal[CoverCallExpressionAndAsyncArrowHead]{?Yield, ?Await}{} \\
    \nonterminal[SuperCall]{?Yield, ?Await}{} \\
    \nonterminal[ImportCall]{?Yield, ?Await}{} \\
    \nonterminal[CallExpression]{}{} \nonterminal[Arguments]{}{} \\
    \nonterminal[CallExpression]{}{} \terminal{[} \nonterminal[Expression]{?Yield, ?Await}{} \terminal{]} \\
    \nonterminal[CallExpression]{}{} \terminal{.} \nonterminal[IdentifierName]{}{} \\
    \nonterminal[CallExpression]{}{} \terminal{.} \nonterminal[PrivateIdentifier]{}{} \\
    \nonterminal[CallExpression]{}{} \nonterminal[TemplateLiteral]{}{}
}

The remaining left recursive productions represent simple lists of items and can be, again, parsed iteratively. These non-terminal are: \nonterminal[FormalParameters]{}{}, \nonterminal[Expression]{}{} (comma operator), \nonterminal[LexicalDeclaration]{}{}, \nonterminal[VariableDeclaration]{}{}, \nonterminal[Arguments]{}{}, and \nonterminal[StatementList]{}{}.


\section{AST}

An abstract syntax tree (AST) is a hierarchical representation of a program's syntactic structure \cite{dragon}. It is a tree generated by the parser, in which every node corresponds to a syntactic construction in the input program.

There are two main kinds of AST nodes -- statements and expressions. Statements describe execution steps and control flow. Expressions may be used as part of statements, and their evaluation results in a value which may be used by another expression. Expressions may also be part of an \textit{expression statement}, which evaluates an expression and ignores its result. Some expressions, such as assignment, may have side effects.

Aside from the two above, a program contains declarations of variables and functions. A function declaration consists of two parts: a signature and a body. Top level function declarations are further processed by the next step.


\section{Generating the intermediate representation}

After parsing the input source code, the compiler transforms parts of the program into the intermediate representation described in Chapter~\ref{ir}.

First, the IR generator iterates over top-level function declarations and creates a list of all possibly compilable functions. In this step, a function is considered compilable if the function signature contains type annotations for all parameters and the return type. The list allows the generator to determine which functions are to be compiled and to insert native calls in relevant places.

Then, function bodies are processed independently on a per-function basis. If compilation of one of these functions fails, the entire compilation process fails with a \texttt{SyntaxError} exception and the input program cannot be executed\footnote{While the program may still be valid and, after stripping type annotations, run by a JavaScript interpreter, the behavior of a compiled function may differ from standard JavaScript. We report an error and let the programmer fix it, rather than make the behavior of a program hard to predict.}.

The AST of each function body is recursively traversed by the compiler and individual nodes are transformed into the intermediate representation described in Chapter \ref{ir}.

The IR generator is implemented as a number of functions which describe the process for different AST nodes. These functions keep track of the context of IR generation, in particular:

\begin{itemize}[noitemsep]
    \item lexical and variable declarations and their scopes,
    \item the active basic block,
    \item break and continue targets, and
    \item the rest of the CFG that is being created.
\end{itemize}

The process of generating the IR can be viewed as two simple cases -- sequential code, and branching code. Generating sequential code is simple, as the respective AST nodes are just replaced by their counterparts consisting of IR instructions, which are emitted to the currently active basic block.

When generating branching statements, multiple new execution paths need to be created. This is achieved by splitting the active basic block into two parts -- entry, and exit. The entry block contains the original code without a terminator instruction while the exit block retains only the terminator instruction. The branching code is then simply emitted into the entry block and a number of newly created blocks for all new paths. All paths that fall through the branching statement meet in the exit block. The exit block then becomes the new active block.


\subsection{Values}

Results of expressions can have different value categories, which allow different operations. Some results may be used as assignment targets (\textit{l-values}), and some may be only used as value operands (\textit{r-values}). When l-values are to be used as value operands in another expression, their value needs to be acquired first.

The type category of the result of some expressions cannot be determined by the node type alone, and depends on the subtree of the node. For example, a \nonterminal[PrimaryExpression]{}{} node may represent a variable (an l-value) or a literal (an r-value). The IR generator therefore first generates the IR code for child nodes and only then determines the value type of the parent.

Sometimes, the final type category of a child node may be incompatible with what the parent node expects. In these cases, the input code contains syntax error, because static semantics of the language are violated. The IR generator fails with \texttt{SyntaxError} exception and aborts compilation.

When a value is saved into an IR register, its reference count needs to be incremented. This operation is implemented by a \texttt{Dup} instruction which is emitted. The reference count needs to be decreased after the value is no longer needed. To avoid analyzing the code and finding the correct places to perform this task, the value is saved using the \texttt{PushUnref} instruction and the reference count is decreased before returning from the function.


\subsection{Examples of IR generation}

The following examples illustrate the process of generating the IR from different AST nodes.

\pagebreak
\paragraph{If} Figure \ref{fig:irgen:if} shows an example of generating the IR for an \texttt{if} statement. The left side shows part of the CFG before processing the \texttt{if} statement -- the active basic block already contains some code (shown by the \texttt{<PrevCode>} line) and has some terminator instruction \texttt{<Terminator1>}.


\begin{enumerate}[noitemsep]
    \item First the active basic block is split into two parts -- \texttt{Pre} and \texttt{Post}. The \texttt{Pre} block contains the original code and remains active, while \texttt{Post} retains only the terminator instruction from the original block.
    \item The code of condition expression is generated into the active (\texttt{Pre}) block.
    \item Blocks \texttt{Then} and \texttt{Else} are created.
    \item \texttt{Branch} instruction is set as the terminator of \texttt{Pre} with targets \texttt{Then} and \texttt{Else}.
    \item The terminator of both blocks is set to \texttt{Jump} (unconditional) with target \texttt{Post}.
    \item The code of each branch is generated by setting the respective block as active and recursively generating the code for the branch statement.
    \item Finally, the \texttt{Post} block is set as the active block and generating of the \texttt{if} statement is finished.
\end{enumerate}

\begin{figure}[!b]
    \begin{minipage}[t]{1\textwidth}
        \begin{minipage}{0.7\textwidth}
            \begin{minted}{typescript}
if (/* <Cond> */) {
    /* <ThenCode> */
} else {
    /* <ElseCode> */
}
            \end{minted}
        \end{minipage}
    \end{minipage}
    \vspace{0.5cm}

    \begin{minipage}[t]{0.49\textwidth}
        \vspace*{0mm}
        \centering
        \includegraphics[scale=0.5, draft=false]{./assets/out/irgen/if-pre.pdf}
    \end{minipage}
    \begin{minipage}[t]{0.49\textwidth}
        \vspace*{0mm}
        \centering
        \includegraphics[scale=0.5, draft=false]{./assets/out/irgen/if-post.pdf}
    \end{minipage}

    \caption{Visualization of IR generation for an \textit{IfStatement}. The left side shows an active basic block before processing the node. The right side shows a CFG generated from the node.}
    \label{fig:irgen:if}
\end{figure}


\pagebreak
\paragraph{While} Figure \ref{fig:irgen:while} shows generating the IR for a \texttt{while} statement. Again, the left side shows the initial active basic block with some code and a terminator instruction. The process is similar to the \texttt{if} statement:

\begin{enumerate}[noitemsep]
    \item The active block is first split into two parts -- \texttt{Pre} and \texttt{Post}. The \texttt{Pre} block contains the original code and \texttt{Post} keeps the terminator instruction.
    \item Two new blocks are created -- \texttt{Cond} and \texttt{Body}.
    \item A \texttt{Jump} instruction is set as the terminator of the \texttt{Pre} block, with target \texttt{Cond}.
    \item A \texttt{Branch} instruction is set as the terminator of the \texttt{Cond} block, with targets \texttt{Body} and \texttt{Post}.
    \item A \texttt{Jump} instruction is set as the terminator of the \texttt{Body} block, with target \texttt{Cond}.
    \item The \texttt{Cond} block is set as the active block and the code for the condition expression is generated.
    \item The \texttt{Body} block is set as the active block and the code for the body statement is generated.
    \item Finally, the \texttt{Post} block is set as the active block and generating of the \texttt{while} statement is finished.
\end{enumerate}

\begin{figure}[!b]
    \begin{minipage}[t]{1\textwidth}
        \begin{minipage}{0.7\textwidth}
            \begin{minted}{typescript}
while (/* <Cond> */) {
    /* <BodyCode> */
}
            \end{minted}
        \end{minipage}
    \end{minipage}
    \vspace{0.5cm}

    \begin{minipage}[t]{0.49\textwidth}
        \vspace*{0mm}
        \centering
        \includegraphics[scale=0.5, draft=false]{./assets/out/irgen/while-pre.pdf}
    \end{minipage}
    \begin{minipage}[t]{0.49\textwidth}
        \vspace*{0mm}
        \centering
        \includegraphics[scale=0.5, draft=false]{./assets/out/irgen/while-post.pdf}
    \end{minipage}

    \caption{Visualization of IR generation for a \textit{WhileStatement}. The left side shows an active basic block before processing the node. The right side shows a CFG generated from the node.}
    \label{fig:irgen:while}
\end{figure}


\pagebreak
\section{MIR backend}

The compiler relies on the MIR backend to perform register allocation, machine code generation, and optimization of the machine code. MIR generates functions with a calling convention native to the platform.

The generated functions are created in two variants -- one with argument types matching the signature of the function, and a companion which receives arguments as an array of JavaScript values. The first of these functions is used when compiled functions interact with each other, while the second is called from stubs provided to QuickJS.

MIR code is generated by simply replacing parts of the IR with the respective MIR representation. Some simple instructions can be replaced by a single MIR instruction, some more complex are replaced by multiple MIR instructions or a function call.

MIR represents a control flow graph, similarly to our IR. The CFG in MIR closely mirrors the CFG of the function in our IR. Some additional blocks are created to represent the function exit and the exception handling code.

The final, native, functions require a runtime context, as described in Section \ref{ir:vm}. The context primarily contains a pointer to an instance of the JavaScript interpreter. It also contains a cleanup pool used by the \texttt{PushUnref} instruction. Lastly, it stores information about whether an exception has been thrown in the function.

Some instructions from our IR and some function calls may cause a JavaScript exception to be thrown. For these, additional checks are inserted into the MIR code. These checks use the exception flag in the runtime context to determine whether an exception has occurred. If an exception occurs, the function returns a default value and the exception flag is set. The caller of the function then checks the exception flag and propagates the exception further up the call stack.

These functions are linked to a single JavaScript runtime and cannot be shared between multiple runtime instances. This allows the functions to use the same runtime context through a constant pointer that is linked to them at compilation.


\subsection{Representation of values}

Values in JavaScript have dynamic type which may fall into one of several categories. In QuickJS, they are represented using a tagged union -- a structure with two 64-bit values representing the tag and a value. In the generated MIR code, this representation is used for the type \texttt{Any}.

For types that require the allocation of additional memory (e.g., \texttt{Object}, \texttt{String}) the tagged union only contains a memory pointer. QuickJS uses reference counting with cycle detection for management of these values. It provides an API for updating the reference count which is used from the generated MIR code.

Because registers in our intermediate representation are strongly typed, we know the type a value has without the need of a tag at runtime. For types other than \texttt{Any}, we can therefore omit the tag and only store their value which fits into a single local variable in MIR.


\section{Stubs}\label{impl:stubs}

After compiling the functions to native code, they need to be made accessible to the interpreter. The compiled functions are therefore wrapped in JavaScript function objects (stubs) which, when called from JavaScript, perform a native call to the compiled function. They are then defined as properties of JavaScript's global object\footnotemark[1] with unique identifiers. This task is performed from C++ code using an API to access the JavaScript runtime.

\footnotetext[1]{In JavaScript, a global object is a special object \texttt{globalThis} which is globally accessible from an execution context (i.e., including different modules). Properties of the global object are a part of the lexical scope, and one of its functions is to provide intrinsic objects and values to the execution environment.}

New JavaScript source code is also generated. In this source code, the original declarations of compiled functions are replaced by references to the stubs, and by comments with the original function's source code.

JavaScript applies declaration and value hoisting\footnotemark[2] on function declarations. This means that in a standard JavaScript program, it is possible to reference and use a function before it is declared. In the newly generated code, we can only replicate this behavior by declaring a new function which would call the stub function. Declaring a new function would result in another level of indirection in every call of this function and add unnecessary overhead.

\footnotetext[2]{Hoisting is a process that allows the use of a binding before its declaration or definition. In the case of function declarations, the value is available as well.}

As maintaining JavaScript's behavior is, in this case, not feasible without modifying the underlying interpreter or additional overhead, we have elected to change the behavior instead. For declaring the aliases for the stub functions, we use a constant lexical declaration (i.e., \texttt{const}). While the runtime semantics is slightly different, its also disallows users from rebinding the same name to a different function (or value). Such behavior would be potentially dangerous, because at compile time functions are statically linked to each other. If the user rebound a function name which was referenced from a different compiled function, they might expect a change in behavior of the compiled function which would, however, not happen. By declaring the function reference as \texttt{const}, such code would result in a syntax error instead.


\section{Interpreter}

The newly generated JavaScript code is then executed by the QuickJS interpreter. The interpreter is used through the Jaculus-machine abstraction layer. After preprocessing the source code, as described in Section \ref{impl:stubs}, and defining new properties of the JavaScript's global object, the new JavaScript code is executed as normal.

\section{Integration with Jaculus-machine}

Jaculus-machine provides a modular way of configuring a JavaScript runtime (called \textit{Machine}) by assembling it from individual modules (called \textit{MFeatures}). The configuration of a selected set of MFeatures is performed at compile time which allows the individual MFeatures to interact with one another without significant overhead caused by dynamic binding.

The compiler implementation is provided as a single MFeature, which provides a method \texttt{eval} for executing input source code. The source code is always processed by the compiler and the generated code is executed.
