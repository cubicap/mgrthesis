\chapter{Implementation}

The compilation pipeline consists of multiple stages. The input is the content of a single TypeScript file, and the output is runnable JavaScript code with stubs referring to generated native functions.


\section{Parser}

The first stage of the compilation pipeline involves parsing the input source code. The TypeScript code is converted to an AST using a recursive descent parser. As the language grammar is left recursive, slight modifications to the grammar and the parser are needed.

Infix binary expressions, in particular, are parsed using a cover grammar:

\begin{minted}{text}
    BinaryExpression -> UnaryExpression [ op BinaryExpression ]*
    op -> infix operator token
\end{minted}

The result is then refined according to the operator precedence and associativity to the correct expression tree using the Shunting Yard algorithm. \todo{}

Another case of left recursion in the grammar are the non-terminals \texttt{MemberExpression} and \texttt{CallExpression}. In both cases, there are multiple expansions with left recursion. Fortunately, all of these expansions represent a repeated application of the same rule. Therefore, they can be parsed iteratively and then transformed into the corresponding syntax tree.

\begin{minted}{text}
    MemberExpression -> SuperProperty
    MemberExpression -> MetaProperty
    MemberExpression -> PrimaryExpression
    MemberExpression -> MemberExpression [ Expression ]
    MemberExpression -> MemberExpression . IdentifierName
    MemberExpression -> MemberExpression . PrivateIdentifier
    MemberExpression -> MemberExpression TemplateLiteral
    MemberExpression -> new MemberExpression Arguments

    CallExpression -> SuperCall
    CallExpression -> ImportCall
    CallExpression -> MemberExpression Arguments
    CallExpression -> CallExpression [ Expression ]
    CallExpression -> CallExpression . IdentifierName
    CallExpression -> CallExpression . PrivateIdentifier
    CallExpression -> CallExpression TemplateLiteral
\end{minted}

The remaining left recursive expansions represent simple lists of items and can be, again, parsed iteratively. The corresponding non-terminal are: \texttt{FormalParameters}, \texttt{Expression} (comma operator), \texttt{LexicalDeclaration}, \texttt{VariableDeclaration}, \texttt{Arguments}, and \texttt{StatementList}.

After generating the AST, the top level statement list is traversed, and function declarations are selected for compilation. The "dependencies" of these functions are also checked to make sure that all function calls stay within the compiled context.


\section{Generating the intermediate representation}

Every AST function consists of two parts: a signature and a body. First, all function signatures are saved so that the generator knows which functions are to be compiled and can insert native calls in relevant places. Then, the function bodies processed independently on a per-function basis. Lastly, functions that failed to compile and functions that require this function (by calling it) are removed and not processed further.

The statements in the function body are recursively traversed by the compiler and the AST is transformed into the intermediate representation described in the previous section.

The IR generator is implemented as a series of overloaded functions for different AST nodes which all operate on the top a shared state object. The state object keeps track of the context of IR generation, in particular, lexical and variable declarations and their scopes, active basic block, break and continue targets, and the whole CFG that is being created.


There are two kinds of AST nodes that generate a part of the IR code - statements and expressions. When an AST node is being processed, all of its children get processed recursively. In the case of expressions, the sub-nodes have a value result, which may be used in some way by the parent node. The results can have different value types can be used in different ways. Some may be used as assignment targets (l-values) and some may be only used as value operands in another expression (r-values). When l-values are to be used as value operands in another expression, they first have to be evaluated (their value is acquired).

Because some AST nodes may result in different value types depending on the whole subtree, a third, union type is used as a result for a later disambiguation. In the implementation, they are represented by the following structures:
\begin{itemize}
\item \texttt{RValue} which represents an r-value,
\item \texttt{LVRef} which represents an l-value, and
\item \texttt{Value} which represents a union of the two above.
\end{itemize}

Sometimes, the final value type of evaluated sub-node can be incompatible with what the parent node expects. In these cases, a syntax error is encountered and the compilation fails.

When a value is saved into an IR register, its reference count needs to be updated. This operation is implemented by a \texttt{Dup} instruction which is emitted. The reference count needs to be decreased after the value is no longer needed. To avoid analyzing the code and finding the correct places to perform this task, the value is saved using the \texttt{PushFree} instruction and the reference count is decreased before returning from the function.

The process of generating the IR code itself can be viewed as two simple cases - linear code, ang branching code. Generating linear code is trivial, as the respective AST nodes are simply replaced by their counterparts consisting of the IR instructions. These can be directly emitted to the currently active basic block. The linear case is illustrated in

When generating branching code, the active basic block has to be split into two parts - entry, and exit. The entry block will contain the original code without the terminator instruction (which will be replaced) while the exit block will retain only the terminator instruction. The branching code can be then simply emitted into the entry block and a number of newly emitted block for all new paths. Paths that do not terminate, or contain non-linear\todo{check} control flow (i.e., \texttt{break} or \texttt{continue} statements) must all converge to the exit block.

\subsection{Examples of IR generation}

\todo{show the IR-gen of some significant AST nodes - linear expression, if, for, short-circuit expression, terminating statement}

\subsection{Runtime requirements of IR function}\todo{find a better title}\label{subsec:irruntime}

There are very few requirements on how a function represented by the intermediate representation should perform. The values in IR registers can be stored in any way, be it in registers, in stack memory, or in some entirely different way. There is also no calling convention.

A function in the IR requires access to the instance of the JavaScript interpreter it runs under. To implement the \texttt{PushFree} instruction, it also needs a secondary stack (or another container) for storing all values that need to be freed.


\subsection{IR interpreter}

An interpreter for the intermediate representation is available as a part of the test suite to allow for testing of the generated IR without compiling it to the native code. It allows for catching IR generation errors early and for avoiding the debugging of the final machine code. The interpreter works directly with the immediate variables and keeps track of their types and reference counts\todo{extend for primitives} during the runtime.


\section{Compiler backend}

The compiler relies on the MIR compiler backend to perform register allocation, code generation, and optimization of the machine code. MIR generates functions with a standard calling convention for a target platform. This calling convention, however, is not compatible with the signature of functions expected by QuickJS. Therefore, for every compiled function, a companion function with a compatible interface is generated.

The MIR code is generated by simply replacing parts of the IR with the respective MIR representation. Some simple instructions can be replaced by a single MIR instruction, some more complex are replaced by multiple MIR instructions or a function call.

The final, native, functions require a runtime context, as described in Section \ref{subsec:irruntime}. The context primarily contains a pointer to an instance of the JavaScript interpreter. It also contains a secondary stack used by the \texttt{PushFree} instruction. Lastly, it holds the values of all string constants used by the compiled functions. As the compiled functions exist only in the context of a single JavaScript interpreter, all functions use the same global context through a global constant pointer that is linked to them.


\section{Stubs}

After compiling the functions to a native code, they need to be provided to the interpreter. This is done by generating a new source code, in which, the function declarations are replaced by stub references and comments with the original function code. The stubs are referenced as an attribute of a global object and are added at the initialization of the interpreter.

Because the functions are statically linked to each other, instead of using a hoisting declaration for these stubs (i.e., \texttt{var}), they are declared using constant lexical declaration (i.e., \texttt{const}). While the runtime semantics is slightly different, its aim is to disallow users to rebind the same name to a different function (or value) and therefore expect a change in behavior of a compiled function.


\section{Interpreter}

The underlying interpreter is unmodified QuickJS with Jaculus-machine abstraction layer.

\section{Integration with Jaculus-machine}

\todo{write}
