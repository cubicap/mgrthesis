\chapter{Motivation}

At Robotárna\todo{?}, we use JavaScript for teaching programming of microcontrollers to beginners. While this approach is unconventional, it has a strong advantage of a very fast code-build-test cycle. The speed of this cycle makes it easier for students to maintain their focus and allows them to experiment with their code.

So far, we have observed success with this approach. We observe students being able to learn basics faster and start working on more interesting projects sooner. However, we have also reached a point where we need to consider the performance of the JavaScript programs.

JavaScript, as an interpreted language, suffers from the inherent flaw of slow execution speed. The slow execution speed is the cost of the ease of use of JavaScript and, in a large part, stems from its dynamic nature and the difficulty to implement more advanced optimizations\todo{vs simple}. Occasionally, more capable students encounter the lower performance of JavaScript as a limitation and have difficulties to implement their ideas.

Because programming in JavaScript as a dynamically typed language can sometimes lead to errors that are difficult to debug, we usually do not write JavaScript directly. Instead, we usually write TypeScript and compile it to JavaScript only before execution.

What is of particular interest, is the fact, that at the time of compilation to JavaScript, type annotations are removed from the input program, and some information about the input program is lost. This fact is unfortunate, because the type information could be used by the runtime environment to perform simple optimizations when loading the program.

In this thesis, we will explore the possibility of using a slightly modified subset of TypeScript as the input of our runtime. We will use the type information to identify segments of code that can be easily compiled to machine code and replace them with their machine code equivalent.

\todo{requirements of the compiler? simplicity, resource constraints...}

We believe, this will lead to a significant improvement in performance of some parts of such input programs that are written in a way, so they can benefit from this optimization.
